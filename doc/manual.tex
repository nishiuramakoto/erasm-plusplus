% -*- mode:latex; -*-
\documentclass{article}
\usepackage[hyphens]{url}
\usepackage{listings}
\usepackage{relsize}
\usepackage{html}

\begin{htmlonly}
  \usepackage{verbatim}
  \providecommand{\lstinputlisting}[2][]{\verbatiminput{#2}}
\end{htmlonly}


%%include lhs2TeX.fmt
%% ODER: format ==         = "\mathrel{==}"
%% ODER: format /=         = "\neq "
%
%
\makeatletter
\@ifundefined{lhs2tex.lhs2tex.sty.read}%
  {\@namedef{lhs2tex.lhs2tex.sty.read}{}%
   \newcommand\SkipToFmtEnd{}%
   \newcommand\EndFmtInput{}%
   \long\def\SkipToFmtEnd#1\EndFmtInput{}%
  }\SkipToFmtEnd

\newcommand\ReadOnlyOnce[1]{\@ifundefined{#1}{\@namedef{#1}{}}\SkipToFmtEnd}
\usepackage{amstext}
\usepackage{amssymb}
\usepackage{stmaryrd}
\DeclareFontFamily{OT1}{cmtex}{}
\DeclareFontShape{OT1}{cmtex}{m}{n}
  {<5><6><7><8>cmtex8
   <9>cmtex9
   <10><10.95><12><14.4><17.28><20.74><24.88>cmtex10}{}
\DeclareFontShape{OT1}{cmtex}{m}{it}
  {<-> ssub * cmtt/m/it}{}
\newcommand{\texfamily}{\fontfamily{cmtex}\selectfont}
\DeclareFontShape{OT1}{cmtt}{bx}{n}
  {<5><6><7><8>cmtt8
   <9>cmbtt9
   <10><10.95><12><14.4><17.28><20.74><24.88>cmbtt10}{}
\DeclareFontShape{OT1}{cmtex}{bx}{n}
  {<-> ssub * cmtt/bx/n}{}
\newcommand{\tex}[1]{\text{\texfamily#1}}	% NEU

\newcommand{\Sp}{\hskip.33334em\relax}


\newcommand{\Conid}[1]{\mathit{#1}}
\newcommand{\Varid}[1]{\mathit{#1}}
\newcommand{\anonymous}{\kern0.06em \vbox{\hrule\@width.5em}}
\newcommand{\plus}{\mathbin{+\!\!\!+}}
\newcommand{\bind}{\mathbin{>\!\!\!>\mkern-6.7mu=}}
\newcommand{\rbind}{\mathbin{=\mkern-6.7mu<\!\!\!<}}% suggested by Neil Mitchell
\newcommand{\sequ}{\mathbin{>\!\!\!>}}
\renewcommand{\leq}{\leqslant}
\renewcommand{\geq}{\geqslant}
\usepackage{polytable}

%mathindent has to be defined
\@ifundefined{mathindent}%
  {\newdimen\mathindent\mathindent\leftmargini}%
  {}%

\def\resethooks{%
  \global\let\SaveRestoreHook\empty
  \global\let\ColumnHook\empty}
\newcommand*{\savecolumns}[1][default]%
  {\g@addto@macro\SaveRestoreHook{\savecolumns[#1]}}
\newcommand*{\restorecolumns}[1][default]%
  {\g@addto@macro\SaveRestoreHook{\restorecolumns[#1]}}
\newcommand*{\aligncolumn}[2]%
  {\g@addto@macro\ColumnHook{\column{#1}{#2}}}

\resethooks

\newcommand{\onelinecommentchars}{\quad-{}- }
\newcommand{\commentbeginchars}{\enskip\{-}
\newcommand{\commentendchars}{-\}\enskip}

\newcommand{\visiblecomments}{%
  \let\onelinecomment=\onelinecommentchars
  \let\commentbegin=\commentbeginchars
  \let\commentend=\commentendchars}

\newcommand{\invisiblecomments}{%
  \let\onelinecomment=\empty
  \let\commentbegin=\empty
  \let\commentend=\empty}

\visiblecomments

\newlength{\blanklineskip}
\setlength{\blanklineskip}{0.66084ex}

\newcommand{\hsindent}[1]{\quad}% default is fixed indentation
\let\hspre\empty
\let\hspost\empty
\newcommand{\NB}{\textbf{NB}}
\newcommand{\Todo}[1]{$\langle$\textbf{To do:}~#1$\rangle$}

\EndFmtInput
\makeatother
%
%
%
%
%
%
% This package provides two environments suitable to take the place
% of hscode, called "plainhscode" and "arrayhscode". 
%
% The plain environment surrounds each code block by vertical space,
% and it uses \abovedisplayskip and \belowdisplayskip to get spacing
% similar to formulas. Note that if these dimensions are changed,
% the spacing around displayed math formulas changes as well.
% All code is indented using \leftskip.
%
% Changed 19.08.2004 to reflect changes in colorcode. Should work with
% CodeGroup.sty.
%
\ReadOnlyOnce{polycode.fmt}%
\makeatletter

\newcommand{\hsnewpar}[1]%
  {{\parskip=0pt\parindent=0pt\par\vskip #1\noindent}}

% can be used, for instance, to redefine the code size, by setting the
% command to \small or something alike
\newcommand{\hscodestyle}{}

% The command \sethscode can be used to switch the code formatting
% behaviour by mapping the hscode environment in the subst directive
% to a new LaTeX environment.

\newcommand{\sethscode}[1]%
  {\expandafter\let\expandafter\hscode\csname #1\endcsname
   \expandafter\let\expandafter\endhscode\csname end#1\endcsname}

% "compatibility" mode restores the non-polycode.fmt layout.

\newenvironment{compathscode}%
  {\par\noindent
   \advance\leftskip\mathindent
   \hscodestyle
   \let\\=\@normalcr
   \let\hspre\(\let\hspost\)%
   \pboxed}%
  {\endpboxed\)%
   \par\noindent
   \ignorespacesafterend}

\newcommand{\compaths}{\sethscode{compathscode}}

% "plain" mode is the proposed default.
% It should now work with \centering.
% This required some changes. The old version
% is still available for reference as oldplainhscode.

\newenvironment{plainhscode}%
  {\hsnewpar\abovedisplayskip
   \advance\leftskip\mathindent
   \hscodestyle
   \let\hspre\(\let\hspost\)%
   \pboxed}%
  {\endpboxed%
   \hsnewpar\belowdisplayskip
   \ignorespacesafterend}

\newenvironment{oldplainhscode}%
  {\hsnewpar\abovedisplayskip
   \advance\leftskip\mathindent
   \hscodestyle
   \let\\=\@normalcr
   \(\pboxed}%
  {\endpboxed\)%
   \hsnewpar\belowdisplayskip
   \ignorespacesafterend}

% Here, we make plainhscode the default environment.

\newcommand{\plainhs}{\sethscode{plainhscode}}
\newcommand{\oldplainhs}{\sethscode{oldplainhscode}}
\plainhs

% The arrayhscode is like plain, but makes use of polytable's
% parray environment which disallows page breaks in code blocks.

\newenvironment{arrayhscode}%
  {\hsnewpar\abovedisplayskip
   \advance\leftskip\mathindent
   \hscodestyle
   \let\\=\@normalcr
   \(\parray}%
  {\endparray\)%
   \hsnewpar\belowdisplayskip
   \ignorespacesafterend}

\newcommand{\arrayhs}{\sethscode{arrayhscode}}

% The mathhscode environment also makes use of polytable's parray 
% environment. It is supposed to be used only inside math mode 
% (I used it to typeset the type rules in my thesis).

\newenvironment{mathhscode}%
  {\parray}{\endparray}

\newcommand{\mathhs}{\sethscode{mathhscode}}

% texths is similar to mathhs, but works in text mode.

\newenvironment{texthscode}%
  {\(\parray}{\endparray\)}

\newcommand{\texths}{\sethscode{texthscode}}

% The framed environment places code in a framed box.

\def\codeframewidth{\arrayrulewidth}
\RequirePackage{calc}

\newenvironment{framedhscode}%
  {\parskip=\abovedisplayskip\par\noindent
   \hscodestyle
   \arrayrulewidth=\codeframewidth
   \tabular{@{}|p{\linewidth-2\arraycolsep-2\arrayrulewidth-2pt}|@{}}%
   \hline\framedhslinecorrect\\{-1.5ex}%
   \let\endoflinesave=\\
   \let\\=\@normalcr
   \(\pboxed}%
  {\endpboxed\)%
   \framedhslinecorrect\endoflinesave{.5ex}\hline
   \endtabular
   \parskip=\belowdisplayskip\par\noindent
   \ignorespacesafterend}

\newcommand{\framedhslinecorrect}[2]%
  {#1[#2]}

\newcommand{\framedhs}{\sethscode{framedhscode}}

% The inlinehscode environment is an experimental environment
% that can be used to typeset displayed code inline.

\newenvironment{inlinehscode}%
  {\(\def\column##1##2{}%
   \let\>\undefined\let\<\undefined\let\\\undefined
   \newcommand\>[1][]{}\newcommand\<[1][]{}\newcommand\\[1][]{}%
   \def\fromto##1##2##3{##3}%
   \def\nextline{}}{\) }%

\newcommand{\inlinehs}{\sethscode{inlinehscode}}

% The joincode environment is a separate environment that
% can be used to surround and thereby connect multiple code
% blocks.

\newenvironment{joincode}%
  {\let\orighscode=\hscode
   \let\origendhscode=\endhscode
   \def\endhscode{\def\hscode{\endgroup\def\@currenvir{hscode}\\}\begingroup}
   %\let\SaveRestoreHook=\empty
   %\let\ColumnHook=\empty
   %\let\resethooks=\empty
   \orighscode\def\hscode{\endgroup\def\@currenvir{hscode}}}%
  {\origendhscode
   \global\let\hscode=\orighscode
   \global\let\endhscode=\origendhscode}%

\makeatother
\EndFmtInput
%
%%include manual.fmt

%c from texinfo.tex
\def\ifmonospace{\ifdim\fontdimen3\font=0pt }

%c C plus plus
\def\C++{%
  \ifmonospace%
  C++%
  \else%
  C\kern-.1667em\raise.30ex\hbox{\smaller{++}}%
  \fi%
  \spacefactor1000 }


\def\ERASM++{%
  \ifmonospace%
  ERASM++%
  \else%
  ERASM\kern-.1667em\raise.30ex\hbox{\smaller{++}}%
  \fi%
  \spacefactor1000 }

\newcommand*{\var}[1]{\mathord{\mathit{#1}}}

\hyphenation{meta-pro-gram-ming}

\begin{document}
\lstset{language=C++}

\title{Erasm++ User's Guide}
\author{Makoto Nishiura \\
\texttt{nishiuramakoto@gmail.com}}

\maketitle

\tableofcontents

\subsection*{Note}
This manual is still only a rough sketch of the library.
I  am working on a better documentation, but if you have encountered a
problem in using the library, please consider reporting it.


\section{Introduction}
\label{sec:introduction}

Erasm++ is a collection of libraries for runtime code generation,
instruction decoding, and metaprogramming. At the core of these
libraries is MetaPrelude, our Haskell-like lazy metaprogramming
library for C++. The emphasis has been on ease-of-use, runtime
performance, and general adaptability at the cost of potentially
increased compile-time resource usage. Our driving motto is this:
\emph{C++ programs can be both much faster and more user-friendly at
  the same time than the functionally equivalent C programs}. As we
will show later in this manual, they can sometimes be an order of
magnitude faster without obscure hand-coded optimizations than some
popular C-based libraries.

\section{ERASM++}
\label{sec:erasm++}

\subsection{Overview}
\label{sec:overview}

ERASM++, the Embedded Runtime Assembler in C++, is an Embedded Domain
Specific Language (EDSL) in C++ for runtime code generation. Its main
strengths are:
\begin{itemize}
\item  Close syntactical similarity with the real assembly
  language.
\item  Complete compile-time syntax checking.
\item Very fast and small code generators.
\item When inlined and optimized by recent compilers, the code
  generator functions compile to just a series of \text{\tt MOV} instructions
  that contain no external references.  In particular, they do not
  require any runtime support.
\item Near full conformance to the C++03 standard subject to the
  limitations concerning the platform's data representations (see
  Section \ref{sec:erasm-limitations} ).
\end{itemize}
Only the Intel 64/IA-32 architectures are supported at this moment. 

\subsection{Basic Examples}
\label{sec:basic-examples}

\subsubsection{32-bit case}
\label{sec:32-bit-case}

First, create a file named \text{\tt demo86\char46{}cpp}\footnote{This program is
  included in \text{\tt PACKAGEROOT\char47{}demo} directory. }  with the following
contents:

\lstinputlisting[language=C++,frame=single]{../demo/demo86.cpp}

\noindent Then, run the following commands on your shell:

\lstset{language=sh}
\begin{tabbing}\tt
~\char36{}~g\char43{}\char43{}~\char45{}DNDEBUG~demo86\char46{}cpp~\char45{}lerasm~\char45{}o~demo86\\
\tt ~\char36{}~\char46{}\char47{}demo86\\
\tt ~\char36{}~\char46{}\char47{}demo86~a~b
\end{tabbing}

\noindent Provided the stack is executable in your operating system by
default, it will print the numbers 3 and 5.

\subsubsection{64-bit case}
\label{sec:64-bit-case}

First, create a file named \text{\tt demo64\char46{}cpp}\footnote{This program is
  included in \text{\tt PACKAGEROOT\char47{}demo} directory. }  with the following
contents:

\lstinputlisting[language=C++,frame=single]{../demo/demo64.cpp}
Then, run the following commands on your shell:
\begin{tabbing}\tt
~\char36{}~g\char43{}\char43{}~\char45{}DNDEBUG~demo64\char46{}cpp~\char45{}lerasm~\char45{}o~demo64\\
\tt ~\char36{}~\char46{}\char47{}demo64\\
\tt ~\char36{}~\char46{}\char47{}demo64~a~b
\end{tabbing}
Provided the stack is executable in your operating system by
default, it will print the numbers 3 and 5.


The namespace \text{\tt erasm\char58{}\char58{}x86\char58{}\char58{}addr32\char58{}\char58{}data32} contains the actual
x86 instruction definitions whose default address and operand sizes
are both 32-bit.  Similarly, the namespace
\text{\tt erasm\char58{}\char58{}x64\char58{}\char58{}addr64\char58{}\char58{}data32} should be used for IA-64 instructions
whose default address and operand sizes are respectively 64-bit and
32-bit. Currently other default address or operand size modes are not
supported.

\subsection{Syntax}
\label{sec:syntax}

In general, the prototype of an ERASM++ instruction is in one of the following forms:

\begin{lstlisting}
int MNEMONIC(code_ptr p);

int MNEMONIC(code_ptr p,
             OP_TYPE op1);

int MNEMONIC(code_ptr p,
             OP_TYPE op1,
             OP_TYPE op2);

int MNEMONIC(code_ptr p,
             OP_TYPE op1,
             OP_TYPE op2,
             OP_TYPE op3);
\end{lstlisting}
\noindent where \text{\tt code\char95{}ptr} is an alias for the type of
pointer to 8-bit unsigned integers on IA-64/IA-32 architectures.

\begin{lstlisting}[caption={Examples of ERASM++ instructions.},frame=single]
p += adc  (p,word_ptr  [edx + esi * _2],ax);
p += and_ (p,dword_ptr [rip + 0x1000],r12d);
p += jmp  (p, 0x10);
p += mov  (p,byte_ptr  [eax*_5],dl);
p += rep_movs (p,dword_ptr.es [di],dword_ptr.gs [si]);
\end{lstlisting}
\begin{lstlisting}[language={[x86masm]Assembler},caption={The corresponding MASM instructions.},frame=single]
adc  WORD PTR  [edx + esi * 2],ax
and DWORD PTR [rip + 0x1000],r12d
jmp  0x10
mov  BYTE PTR  [eax*5],dl
rep movs DWORD PTR es:[di],DWORD PTR gs:[si]
\end{lstlisting}

The operational meaning of each instruction is always the same: it
encodes the instruction at the location specified by the first
argument, and returns the number of bytes encoded. If the target
address is writable, it is guaranteed that either it succeeds to
encode the instruction, or the instruction is syntactically ill-formed
and the compilation fails. If the destination address is not writable,
the behavior is undefined. In particular, it never throws a (C++)
exception.


{\bf MNEMONIC} is generally one of the instruction mnemonics as
defined in the Intel Instruction Set Reference~\cite{intel:insn_ref},
with each letter converted to lower-case. However, there are some
exceptions:
\begin{itemize}
\item If the mnemonic is reserved as a keyword in C++ , an underscore
  is appended, as in \text{\tt int\char95{}},\text{\tt xor\char95{}}, etc.
\item If the instruction requires a mandatory prefix such as
  REP, REPZ etc., then the prefix and the mnemonic of the
  instruction are joined by an underscore, as in \text{\tt repe\char95{}scas}.
\item Some instructions are ambiguous and cannot be fully specified
  solely by the combination of the mnemonic and operand types. In such
  a case, we resort to rather ad-hoc case-by-case disambiguation
  schemes such as:
\begin{lstlisting}[frame=single]
p +=      rep_movs(p,byte_ptr[rdi],byte_ptr[rsi]);
p += rexw_rep_movs(p,byte_ptr[rdi],byte_ptr[rsi]);    
\end{lstlisting}
  Here, \text{\tt rep\char95{}movs} uses \text{\tt ecx} as the count register, while \text{\tt rexw\char95{}rep\char95{}movs} uses \text{\tt rcx}.
  We are searching for a less ad-hoc disambiguation method, but we would also
  like the form of these instructions to be easily guessable from the definitions
  in Intel Instruction Set Reference~\cite{intel:insn_ref}.
\end{itemize}

{\bf OP{\_}TYPE} is the type of an instruction operand
which is one of the following objects:
\begin{description}
\item[Registers.] \text{\tt dh},\text{\tt ax},\text{\tt edx},\text{\tt xmm0},\text{\tt st3},\text{\tt rax},\text{\tt r8},\text{\tt r9b},\text{\tt sil}, etc.
\item[Memory references.]  Examples:
  \begin{itemize}
  \item  \text{\tt byte\char95{}ptr~\char91{}esp~\char43{}~ebp~\char42{}~\char95{}2~\char43{}~4\char93{}}
  \item  \text{\tt dword\char95{}ptr~\char91{}ebx~\char42{}~\char95{}8~\char43{}~\char95{}4\char93{}}
  \item  \text{\tt xmmword\char95{}ptr~\char46{}~gs\char91{}bx\char43{}di\char93{}}
  \end{itemize}

\item[Immediates.] \text{\tt \char40{}int32\char95{}t\char41{}0},\text{\tt \char40{}int8\char95{}t\char41{}1},\text{\tt \char95{}1},\text{\tt \char95{}2}, etc. 
\end{description}

The static constants \text{\tt \char95{}0}-\text{\tt \char95{}9} can be used as a shorthand for small
8-bit integers. By nature, assembly instructions are quite picky about
the exact operand types, and it is very often the case that either
using static constants or casting explicitly is required to avoid the
ambiguity.

Note that the scale coefficient of the index register (e.g. \text{\tt \char95{}2} in
the expression \text{\tt \char91{}edx~\char43{}~\char95{}2~\char42{}~ecx\char93{}}) must be specified with the static
constants \text{\tt \char95{}1} - \text{\tt \char95{}9}. This inconvenience results from the need to
compute the necessary data statically, and will not change.

Please see Intel  Instruction Set Reference~\cite{intel:insn_ref} for
the details of instruction definitions.

For the examples of syntactically valid instructions, see

\begin{itemize}
\item \text{\tt src\char47{}erasm\char47{}x64\char95{}assembler\char95{}manual\char95{}test\char46{}cpp}
\item \text{\tt src\char47{}erasm\char47{}x64\char95{}assembler\char95{}auto\char95{}test\char46{}cpp}
\item \text{\tt src\char47{}erasm\char47{}x86\char95{}assembler\char95{}manual\char95{}test\char46{}cpp}
\item \text{\tt src\char47{}erasm\char47{}x86\char95{}assembler\char95{}auto\char95{}test\char46{}cpp}
\end{itemize}

For the examples of ill-formed instructions, see

\begin{itemize}
\item \text{\tt src\char47{}erasm\char47{}x64\char95{}assembler\char95{}illegal\char95{}code\char46{}cpp}
\item \text{\tt src\char47{}erasm\char47{}x86\char95{}assembler\char95{}illegal\char95{}code\char46{}cpp}
\end{itemize}

\subsection{Performance}

If all the required functions are inlined(see below), ERASM++
instructions will be compiled to a series of simple memory copy
instructions with no external references(except possibly the reference
to the execution stack).

Let us give an example. On a 32-bit Pentium machine, g++ 4.6.1 compiles
(with the option -O3 to ensure inlining) the following code:

\begin{lstlisting}[frame=single]
inline int adc (code_ptr p, 
                const ByteReg86 & op1,
                const BytePtr86 & op2)
{
  // details omitted
}

extern "C"  int code_test(code_ptr p)
{ return adc(p ,cl,byte_ptr [eax+ebp*_4+12345678]);  }
\end{lstlisting}
to:
\begin{lstlisting}[language={[x86masm]Assembler},frame=single]
_code_test:
sub	esp, 32
mov	eax, DWORD PTR [esp+36]
mov	BYTE PTR [esp+5], -88
mov	DWORD PTR [esp+6], 12345678
mov	BYTE PTR [esp+4], -116
mov	BYTE PTR [eax], 103
mov	BYTE PTR [eax+1], 18
mov	edx, DWORD PTR [esp+4]
mov	DWORD PTR [eax+2], edx
mov	edx, DWORD PTR [esp+8]
mov	WORD PTR [eax+6], dx
mov	eax, 8
add	esp, 32
ret
\end{lstlisting}

Although the above code may not be the fastest possible, we observe the
following very nice properties:

\begin{description}
\item[No branches.] Conditional or not, branches always consume
  precious processor resources~\cite{intel:opt_ref}, so their absence
  is always beneficial for performance reasons. Furthermore, in kernel
  spaces, there is a greater risk that the target address is not
  accessible at all (e.g. the corresponding section may have been
  swapped out and the code runs in a high priority mode.)

\item[No external references.] This is also a very good property for
  the similar reasons. See~\cite{intel:opt_ref} for the details.

\end{description}

Although there are strong arguments against introducing C++ code
into the operating system kernel
spaces~\cite{linux:cpp_kernel}\cite{ms:cpp_kernel}, it should be clear
that ERASM++ \textit{per se} does not possess any such weakness that
makes it hard to adopt C++ code into the kernel space.

Currently the main ERASM++ encoder functions are all compiled in a
seperate module for the sake of compile\-time performance. 

See \text{\tt src\char47{}test\char95{}asm\char95{}performance\char46{}cpp} for how to inline the required
functions.

\subsection{Limitations}
\label{sec:erasm-limitations}

\paragraph{Compile-time Performance}
\label{sec:comp-time-perf}

By far the biggest limitation of ERASM++ is its compile-time resource
usage. Because of the heavy use of template metaprogramming
techniques, the exact compile-time behaviour greatly varies from one
compiler version to another.

For example, while g++-4.5.4 compiles a source file which consists of
about 2500 instructions just fine with memory usage $\sim$500MB,
g++-4.6.1 consumes all available user-space (typically up to 2GB in
32-bit systems) and eventually gives up.

\paragraph{Error Messages}
\label{sec:error-messages}

Another disadvantage of using ERASM++ is its often cryptic error
messages.  This is inherently an implementation-dependant problem, and
is impossible to solve generally. Current implementation only makes
some effort to increase readability for g++.

Here's an example (manually formatted) error message from
g++. Given the code
\begin{lstlisting}[frame=single]
cvtpd2pi(p,mm1,xmmword_ptr[rax+ rsp*_2]);  
\end{lstlisting}
g++-4.6.1 prints the following message:
\begin{lstlisting}[emph={Error_ESP_and_RSP_cannot_be_used_as_the_index_register},emphstyle=\underbar]
./erasm/meta_prelude_core.hpp:829:128: 
error: no type named 'result' in 
'struct erasm::prelude::eval<
erasm::prelude::if_<
 erasm::prelude::error<
  erasm::x64::
  Error_ESP_and_RSP_cannot_be_used_as_the_index_register,
   erasm::prelude::Term<erasm::prelude::Int<4>, 
    erasm::prelude::times<erasm::prelude::Int<2>,
                          erasm::prelude::Int<1> > >,
   erasm::prelude::Term<erasm::prelude::Int<0>, 
                        erasm::prelude::Int<1> >,
   void, void>, 
 erasm::x64::PtrRex<erasm::x64::XmmWord>, 
 erasm::x64::PtrNoRex<erasm::x64::XmmWord> > >'
...
\end{lstlisting}
Note the underlined identifier above, which correctly spots the problem.

Unfortunately, Microsoft's \text{\tt cl\char46{}exe} tries to outsmart our effort and emits
more obscure error messages.

\paragraph{Labels}
\label{sec:labels}

We have not implemented labels yet. It is necessary to maintain code positions manually
if back-patching or relocations are required.

\paragraph{Standard Conformance}
\label{sec:standard-conformance}

ERASM++ declares static constants \text{\tt \char95{}1},\text{\tt \char95{}2}, etc. in one of its namespaces. This
is potentially problematic as the standard specifies that any identifier that
begins with an underscore is reserved for the implementation in the global
namespace.

Finally, ERASM++ cannot be fully standard conformant because it must
be dependent on the platform-specific data representations. For
example, the Intel 64/IA-32 related code relies on the availability of
8-, 16-, 32-, and 64-bit integers, the availability of \text{\tt \char35{}pragma~pack}
compiler directive, and the little-endianess of the platform.
However, if the compiler does support the platform, there should be no
problem compiling ERASM++ as long as the compiler is reasonably
standard conformant.

\section{GenericDsm}
\label{sec:gener-dsm}

\subsection{Overview}
\label{sec:overview-1}

GenericDsm is a fast and generic instruction decoder library
that supports pattern-match like interface which resembles that of
modern functional languages such as Haskell. Features include:
\begin{description}
\item [Genericity.] Since the decoder function is a template function
  parametrized by the user-supplied set of actions, it can adapt to
  any kind of needs without sacrificing performance.

\item[Speed.] Genericity allows specializing to one's custom needs. In
  some cases, this leads to an instruction decoder which is more than 
  an order  of magnitude faster than non-generic instruction decoder
  libraries such as libopcodes (a library which is ubiquitous in
  GNU/Linux systems).

\item[Pattern-matching against instructions.] Not only does it lead to
  cleaner code, it also contributes to reducing common overheads in
  many existing disassembler libraries. For example, it allows using
  static compile-time predicates on a set of instructions and
  operands, avoiding the common runtime tests in C-based libraries.

\item[No dependence on runtime system.] Just like ERASM++, the main
  decoder function in GenericDsm itself does not depend on other
  libraries including STL or even the C runtime libraries, except some
  Boost's header-only libraries. Furthermore, it does not throw
  exceptions, rely on vtables, or allocate memory outside the
  stack. This allows GenericDsm to be used in most demanding
  circumstances.

\item [Reentrant.] Since the decoder function itself only uses local
  stack-allocated variables, it is fully reentrant and thread-safe as
  long as the user-supplied actions are. (Assuming the decoded
  instructions are constant, of course.)

\end{description}

Only Intel 64/IA-32 architectures are supported. Also, there is a
considerable compile-time overhead using GenericDsm as is
common in all C++ programs with heavy use of metaprogramming and
inlining.

\subsection{Basic Examples}
\label{sec:basic-examples-1}

\subsubsection{32-bit case}
\label{sec:32-bit-case-1}

First, create a file named \text{\tt dsm86\char46{}cpp}\footnote{Included in
  PACKAGEROOT/demo directory.} with the following contents:

\lstinputlisting[language=C++,frame=single]{../demo/dsm86.cpp}

Then, run the following commands on your shell:

\lstset{language=sh}
\begin{tabbing}\tt
~\char36{}~g\char43{}\char43{}~\char45{}DNDEBUG~\char45{}O3~dsm86\char46{}cpp~\char45{}lerasm\char95{}dsm~\char45{}o~dsm86\\
\tt ~\char36{}~\char46{}\char47{}dsm86
\end{tabbing}

On a 32-bit system, the output will look like the following:

\lstinputlisting[language=C++]{../demo/dsm86.out}

\subsubsection{64-bit case}
\label{sec:64-bit-case-1}

First, create a file named \text{\tt dsm64\char46{}cpp} with the following contents:

\lstinputlisting[language=C++,frame=single]{../demo/dsm64.cpp}

Then, run the following commands on your shell:

\lstset{language=sh}
\begin{tabbing}\tt
~\char36{}~g\char43{}\char43{}~\char45{}DNDEBUG~\char45{}O3~dsm64\char46{}cpp~\char45{}lerasm\char95{}dsm~\char45{}o~dsm64\\
\tt ~\char36{}~\char46{}\char47{}dsm64
\end{tabbing}

On a 64-bit system, the output will look like the following:

\lstinputlisting[language=C++]{../demo/dsm64.out}

Note that the directive \text{\tt using~SimpleDsm\char58{}\char58{}action} in those two
examples is mandatory unless your compiler employs a non-standard name
resolution algorithm.


As we will discuss shortly, the above apparently quite simple examples
already give \emph{very fast} disassemblers. Indeed, they are more
than four times as fast as libopcodes based disassemblers on one of our
systems (Windows 7 32-bit, Pentium Dual E2220 2.40GHz) \emph{including
  IO}.

\subsection{Overview of the API}
\label{sec:overview-api}

In this section, we outline the current library interface. Note that
the current implementation is still in its alpha stage, and the interface
\emph{will} change as we receive feedbacks from users/testers. 

Here is the prototypes of the two most important functions in the library:
\begin{lstlisting}[frame=single]
namespace erasm { namespace x64 { 
namespace addr64 { namespace data32 {

    template<class Action>
    const_code_ptr
    decode(const_code_ptr decode_start,
           Action & action);

}}}}

namespace erasm { namespace x86 { 
namespace addr32 { namespace data32 {

    template<class Action>
    const_code_ptr
    decode(const_code_ptr decode_start,
           Action & action);

}}}}
\end{lstlisting}

The functions take two arguments, the first of which denotes the address at which
the disassembly starts, and the second what action it should take for each
decoded instruction. It returns the address the decoding has ended.

To describe the condition under which the class {\bf Action} is to be a
valid argument type, recall the prototype of an $n$-ary ($n = 0,1,2,3$)
ERASM++ instruction:
\begin{lstlisting}[mathescape,frame=single]
int $\var{mnemonic}$(code_ptr_type p,
            $\var{OP{\_}TYPE}_{0}$ $\var{op}_{0}$, 
            .. ,
            $\var{OP{\_}TYPE}_{n-1}$ $\var{op}_{n-1}$);
\end{lstlisting}

For each \textit{mnemonic}, we define the corresponding class
\textit{Mnemonic}, a class whose name is the same as \textit{mnemonic}
except the first letter is upper-case. The class \textit{Mnemonic} is
a non-virtual subclass of \text{\tt InstructionData}, which we will explain
later.

Now, for an object \text{\tt act} of type \text{\tt Action} to be a valid argument
for \text{\tt decode}, the following expression must be valid for
any object \text{\tt insn} of type \textit{Mnemonic}:

\begin{lstlisting}[mathescape,frame=single]
std::pair<const_code_ptr,ACTION_TYPE> ret 
     = act.action(insn,$op_{0}$,..,$op_{n-1}$);
\end{lstlisting}
Here, \text{\tt ret\char46{}second} is either:
\begin{description}
\item[\text{\tt ACTION\char95{}FINISH}]   , to instruct the function to return (with value \text{\tt ret\char46{}first}), or
\item[\text{\tt ACTION\char95{}CONTINUE}] , to instruct the function to continue decoding at the address specified by
\text{\tt ret\char46{}first}.
\end{description}



\noindent Furthermore, the following expression must be valid:
\begin{lstlisting}[frame=single]
const_code_ptr ret = act.error(decode_data);
\end{lstlisting}
\noindent where \text{\tt decode\char95{}data} is an object of class \text{\tt InstructionData} and
contains the location of decode error.

The following prototype illustrates the sufficient 
condition to be the type of a valid argument for \text{\tt decode}.

\begin{lstlisting}[frame=single]
typedef std::pair<const_code_ptr,ACTION_TYPE> 
        action_result_type;

struct Action
{
   template<class Insn>
   action_result_type
   action(const Insn& insn);

   template<class Insn,class Op1>
   action_result_type
   action(const Insn& insn,
	  const Op1& op1);

   template<class Insn,class Op1,class Op2>
   action_result_type
   action(const Insn& insn,
	  const Op1& op1,
	  const Op2& op2);

   template<class Insn,class Op1,class Op2,class Op3>
   action_result_type
   action(const Insn& insn,
	  const Op1& op1,
	  const Op2& op2,
	  const Op3& op3);
   
   const_code_ptr_type        
   error(const InstructionData& params);

};
\end{lstlisting}

For example, the following code defines a simple instruction counter
that only considers \text{\tt ret} and \text{\tt jmp} instructions:

\lstinputlisting[language=C++,frame=single]{../demo/dsm86_counter.cpp}

Note how both direct and indirect \text{\tt jmp} instructions are matched by the following 
member declaration:

\begin{lstlisting}[frame=single]
template <class Op> 
action_result_type action(const Jmp& insn,const Op& op);
\end{lstlisting}

If the user wants to treat the 8-bit direct \text{\tt jmp} instructions specially,
it is just as simple as adding the following member definition:
\begin{lstlisting}[frame=single]
  action_result_type action(const Jmp& insn,rel8_t op1);
\end{lstlisting}

Each instruction class is derived from \text{\tt InstructionData}, which
is defined in
\text{\tt erasm\char47{}x64\char95{}instruction\char95{}definition\char95{}common\char46{}hpp}. (Caution: the
interface of this class \textit{will} change.)  Also, for each
instruction class a static member \text{\tt mnemonic} of type \text{\tt const~char~\char42{}~const} is defined and holds the obvious value.


Finally, for any operand object \text{\tt x} and for any ostream object
\text{\tt os} , the expression \text{\tt os~\char60{}\char60{}~x} is guaranteed to be valid
and prints appropriate information.

\subsection{Performance}
\label{sec:performance}

We have compared the performance of GenericDsm and several free
decoder libraries for various tasks. The tasks were conducted 
on the following two machines:
\begin{description}
\item[Machine 1]  Pentium Dual E2220 2.40GHz, Windows 7 32-bit.
\item[Machine 2]  Celeron M 1.3GHz, Linux 32-bit.
\end{description}
The results are summarized in
Table \ref{tab:machine1} and Table \ref{tab:machine2}.

\paragraph{Libraries}
\label{sec:libraries}

The libraries used for the comparison are the following:
\begin{description}
\item[GenericDsm (GD)] version 0.10, compiled with:
  \begin{itemize}
  \item Mingw gcc 4.6.1 on Machine 1, option -O3 
  \item Debian gcc 4.6.1-4 on Machine 2, option -O3 
  \end{itemize}
  
\item[DynamoRIO \cite{dynamorio} (DR).] The following binary distributions were used:
  \begin{itemize}
  \item DynamoRIO-Windows-3.1.0-4 on Machine 1,
  \item DynamoRIO-Linux-3.1.0-4 on Machine 2.
  \end{itemize}

\item[udis86 \cite{udis86}]  version 1.7, compiled with:
  \begin{itemize} 
  \item Mingw gcc 4.6.1 on Machine 1, option -O3 
  \item Debian gcc 4.6.1-4 on Machine 2, option -O3 
  \end{itemize}

\item[libopcodes \cite{binutils}.] The following binary distributions were used:
  \begin{itemize}
  \item  Mingw GNU binutils version 2.21.53.20110804 on Machine 1
  \item  Debian GNU binutils version 2.21.52.20110606 on Machine 2
  \end{itemize}

\end{description}

\paragraph{Tasks}
\label{sec:tasks}

All the tasks were conducted ten times on the same contiguous block of
instructions.  This block consisted of the equal numbers of all
supported instructions in IA-32 mode, totaling 2150000 instructions
that consumed about 10M bytes of memory.

The tasks conducted are the following:
\begin{itemize}
\item Counting the number of instructions in the block.
\item Counting the number of CTIs, the Control Transfer Instructions,
  which include JMP, CALL, Jcc, LOOPcc and other instructions that affect the
  instruction pointer register.
\item Printing disassembly of the same block in Format A (see Table \ref{tab:formats}).
\item Printing disassembly of the same block in Format B (see Table \ref{tab:formats}).
\item Printing disassembly of the same block in Format C (see Table \ref{tab:formats}).
\end{itemize}

\noindent In order to give a rough estimate of the significance of IO
overhead in these tasks, we have compared the performance of
disassembly in three different formats.  This was necessary because
the formatted IO is hard-coded in the decoder functions of some of the
libraries compared.

\begin{table}[p]
  \centering
  \begin{tabular}[center]{l|l|l|l|l}
    Task                   & GD    & DR   & udis86 & libopcodes \\
    Count Instructions     & 0.19 & 0.11  & 5.00  & 10.55       \\
    Count CTIs             & 0.19 & 1.34  & N/A   & N/A         \\
    Disassembly (Format A) & 2.28 & N/A   & 6.58  & N/A         \\
    Disassembly (Format B) & 2.89 & 14.01 & 7.08  & N/A         \\
    Disassembly (Format C) & 4.10 & N/A   & 8.75  & 18.55       \\
  \end{tabular}
  \caption{Results on Machine 1 (Pentium Dual E2220 2.40GHz, Windows 7 32-bit).
    The numbers indicate the average elapsed times in seconds.}
  \label{tab:machine1}

\end{table}

\begin{table}[p]
  \centering
  \begin{tabular}[center]{l|l|l|l|l}
    Task                   & GD    & DR    & udis86 & libopcodes   \\
    Count instructions     & 0.33  & 0.21  & 9.34   & 4.99         \\
    Count CTIs             & 0.31  & 7.86  & N/A    & N/A          \\
    Disassembly (Format A) & 3.99  & N/A   & 10.56  & N/A          \\
    Disassembly (Format B) & 4.74  & 29.94 & 11.47  & N/A          \\
    Disassembly (Format C) & 6.83  & N/A   & 13.79  & 14.44        \\
  \end{tabular}
  \caption{Results on Machine 2 (Celeron M 1.3GHz, Linux 32-bit).
    The numbers indicate the average elapsed times in seconds.}
  \label{tab:machine2}
\end{table}

\begin{table}[p]
  \centering
  \begin{tabular}[center]{l|l l l l}
    & Opcode & Address  & Length & Instruction \\
    \hline
    Format A:&        &          &        & adc al,0x12 \\
    Format B:&        & 08159c18 &        & adc al,0x12 \\
    Format C:& 1412   & 08159c18 & 2      & adc al,0x12 \\
  \end{tabular}
  \caption{Disassembly formats used for the comparison.}
  \label{tab:formats}
\end{table}


\paragraph{Conclusions}
\label{sec:conclusions}

\begin{itemize}
\item DynamoRIO uses adaptive level-of-detail instruction
  representation~\cite{bruening:dynamorio} and provides specialized
  decoder functions for each representation. In particular, it
  provides a very fast instruction decoder \text{\tt decode\char95{}next\char95{}pc} which
  calculates only the next instruction address. This was the only
  function that outperformed GenericDsm in this comparison. 

\item In all other tasks, GenericDsm was 2-70 times faster than other
  libraries.

\item The three libraries we have compared against use \text{\tt printf} or its
  variants for formatted output, whereas GenericDSM uses a custom
  ostream object \text{\tt erasm\char58{}\char58{}cout} which turned out be slightly faster in
  some cases than printf or its variants.

\item In libopcodes and udis86, the formatted IO code is tightly
  coupled in the decoder function and there seems to be no easy way to
  measure overheads related to formatted IO separately.  This means
  that even mere instruction counting involves some overhead related
  to formatted IO in these libraries.  In contrast, GenericDsm could
  easily adapt to all the tasks with no unnecessary overheads.
\end{itemize}

\subsection{Limitations}
\label{sec:limitations}

\begin{itemize}
\item Slow compilation.
\item Can cause code bloat if instantiated to too many actions.
\item No support for symbol resolution.
\item No support for detecting the instruction's effects on the \text{\tt eflags} register. 
\item Only relatively recent, reasonably standard conformant compilers
  can be used.
\item Compilation errors can cause long and unfriendly error messages.
\item The library interface is still unstable.
\end{itemize}

\section{License}
Copyright 2010--2012 Makoto Nishiura.

Erasm++ is free software; you can redistribute it and/or modify it under
the terms of the GNU General Public License as published by the Free
Software Foundation; either version 3, or (at your option) any later
version.

Erasm++ is distributed in the hope that it will be useful, but WITHOUT ANY
WARRANTY; without even the implied warranty of MERCHANTABILITY or
FITNESS FOR A PARTICULAR PURPOSE.  See the GNU General Public License
for more details.

You should have received a copy of the GNU General Public License
along with Erasm++; see the file COPYING.  If not see
\url{http://www.gnu.org/licenses/}.

\bibliographystyle{plain}
\bibliography{manual}

\end{document}

